\documentclass[a4paper]{article}

\usepackage[english]{babel}
\usepackage[utf8]{inputenc}
\usepackage{amsmath}
\usepackage{graphicx}
\usepackage[colorinlistoftodos]{todonotes}

\usepackage{theorem}
\usepackage{amssymb}

\usepackage{hyperref}

\newenvironment{proof}{{\bf Proof:  }}{\hfill\rule{2mm}{2mm}}

\newtheorem{fact}{Fact}[section]
\newtheorem{lemma}[fact]{Lemma}
\newtheorem{theorem}[fact]{Theorem}
\newtheorem{definition}[fact]{Definition}
\newtheorem{corollary}[fact]{Corollary}
\newtheorem{proposition}[fact]{Proposition}
\newtheorem{claim}[fact]{Claim}
\newtheorem{exercise}[fact]{Exercise}



\title{Extended Avanti's CLT Theorem}

\author{Runze}

\date{\today}

\begin{document}
\maketitle

All notations in this writeup are according to the latest version of \textbf{\textit{A limit theorem for scaled eigenvectors of random dot product graphs}}.




\begin{corollary}
\label{cor:Xsum}
Let $X_1, \cdots, X_n$ be mean-zero independent random matrices, defined on a common probability space $(\Omega, \mathcal{F}, \mathbb{P})$, with values in $\mathbb{C}_{\mathrm{Herm}}^{d \times d}$ and such that there exists a $M > 0$ with $\|X_i\| \le M$ almost surely for all $1 \le i \le m$.
Define $\sigma^2 \equiv \lambda_{\max} \left( \sum_{i=1}^n \mathbb{E}[X_i^2] \right)$. Then for all $t\ge 0$,
\[
	\mathbb{P} \left( \lambda_{\max} \left( \sum_{i=1}^n X_i \right) \ge t \right) \le d e^{- \frac{t^2}{8 \sigma^2 + 4 M t}},
\]
and
\[
	\mathbb{P} \left( \left\| \sum_{i=1}^n X_i \right\| \ge t \right) \le 2d e^{- \frac{t^2}{8 \sigma^2 + 4 M t}}.
\]
\end{corollary}
\textbf{Remark:} See Corollary 7.1 in \href{url}{http://arxiv.org/pdf/0911.0600v2.pdf}.





\begin{theorem}
\label{thm:AbarDiff}
Let $P:[n]^2 \rightarrow [0,1]$ be symmetric, and $A^{(1)}, \cdots, A^{(m)} \stackrel{iid}{\sim} \mathrm{Bern}(P)$ also be symmetric. Define $\bar{A} = \frac{1}{m} \sum_{t = 1}^m A^{(t)}$, then for any constant $c > 0$ there exists another constant $n_0(c)$, independent of $n$ or $P$, such that if $n > n_0$, then for all $\eta$ satisfying $n^{-c} \le \eta \le 1/2$,
\[
	\mathbb{P} \left( \| \bar{A} - P \| \le 4 \sqrt{n \ln(n/\eta)} \right) \ge 1 - \eta.
\]
\end{theorem}
\textbf{Remark:} This is the extended version of Theorem 3.1 in \href{url}{http://arxiv.org/pdf/0911.0600v2.pdf}.
\begin{proof}
Let $\{e_i\}_{i=1}^n$ be the canonical basis for $\mathbb{R}^n$. For each $1 \le i, j \le n$, define a corresponding matrix $G_{ij}$:
\[
    G_{ij} \equiv \left\{
    \begin{array}{l l}
        e_i e_j^T + e_j e_i^T, \quad & i \ne j;\\
        e_i e_i^T, \quad & i = j.
    \end{array}
    \right.
\]
Thus $A^{(t)} = \sum_{1 \le i \le j \le n} A^{(t)}_{ij} G_{ij}$ for $1 \le t \le m$, $\bar{A} = \sum_{1 \le i \le j \le n} \frac{1}{m} \sum_{t = 1}^m A^{(t)}_{ij} G_{ij}$ and $P = \sum_{1 \le i \le j \le n} P_{ij} G_{ij}$. Then we have $\bar{A} - P = \sum_{1 \le i \le j \le n} X_{ij}$, where $X_{ij} \equiv \left( \frac{1}{m} \sum_{t=1}^m A_{ij}^{(t)} - P_{ij} \right) G_{ij}$, $1 \le i \le j \le n$.

Notice that random matrices $X_{ij}$ are independent and have mean zero. Also,
\[
	\|X_{ij}\| \le \|A_{ij}\| = 1
\]
as the eigenvalues of $A_{ij}$ are always contained in the set $\{-1, 0, 1\}$. Thus the assumptions of Corollary \ref{cor:Xsum} apply with $M=1$.

Since
\begin{align*}
	\mathbb{E}[X_{ij}^2] = & Var(X_{ij}) \\
    = &	Var \left( \frac{1}{m} \sum_{t=1}^m A_{ij}^{(t)} - P_{ij} \right) G_{ij}^2 \\
    = &	Var \left( \frac{1}{m} \sum_{t=1}^m A_{ij}^{(t)} \right) G_{ij}^2 \\
    = & \frac{1}{m} Var(A_{ij}^{(t)}) G_{ij}^2 \\
    = & \frac{1}{m} P_{ij} (1-P_{ij}) G_{ij}^2
\end{align*}
and
\[
    G_{ij}^2 \equiv \left\{
    \begin{array}{l l}
        e_i e_i^T + e_j e_j^T, \quad & i \ne j;\\
        e_i e_i^T, \quad & i = j.
    \end{array}
    \right.
\]
Therefore,
\begin{align*}
	\sum_{i \le j} \mathbb{E}[X_{ij}^2] & = \sum_i \frac{1}{m} P_{ii} (1-P_{ii}) e_i e_i^T +
\sum_{i<j} \frac{1}{m} P_{ij} (1-P_{ij}) (e_i e_i^T + e_j e_j^T) \\
	& = \sum_{i=1}^n \left( \frac{1}{m} \sum_{j=1}^n P_{ij} (1-P_{ij}) \right) e_i e_i^T.
\end{align*}
This is a diagonal matrix. Define $\Delta \equiv \max_{i \in [n]} d_P(i)$, then its largest eigenvalue is at most
\[
	\max_{i \in [n]} \left( \frac{1}{m} \sum_{j=1}^n P_{ij} (1-P_{ij}) \right)
    \le \max_{i \in [n]} \frac{1}{m} \sum_{j=1}^n P_{ij} = \Delta/m.
\]

Apply Corollary \ref{cor:Xsum} with $\sigma^2 = \Delta/m$ and $M = 1$, we get for any $t>0$, $m \ge 1$,
\begin{equation}
\label{eqn:t}
	\mathbb{P}(\| \bar{A} - P \| \ge t) \le 2n e^{- \frac{t^2}{8 \Delta/m + 4t}}
    \le 2n e^{- \frac{t^2}{8n + 4t}}.
\end{equation}

Now let $c > 0$ be given and assume $n^{-c} \le \eta \le 1/2$. Then there exists a $n_0(c)$ independent of $n$ and $P$ such that whenever $n > n_0(c)$,
\[
	t = 4 \sqrt{n \ln(2n/\eta)} \le 2 n.
\]

Plugging this $t$ into Equation (\ref{eqn:t}), we get
\[
	\mathbb{P}(\| \bar{A} - P \| \ge 4 \sqrt{n \ln(2n/\eta)})
    \le 2n e^{-\frac{t^2}{16 n}} = 2n e^{-\frac{16 n \ln(2n/\eta)}{16 n}} = \eta.
\]
\end{proof}







\begin{proposition}
\label{prop:AbarSquareDiff}
It holds with probability greater than $1 - \frac{2}{n^2}$ that
\[
      \| \bar{A}^2 - P^2 \|_F \le \sqrt{2 n^3 \log n}.
\]
\end{proposition}
\textbf{Remark:} This is Proposition 4.2 in \href{url}{http://arxiv.org/pdf/1207.6745.pdf}.

\noindent
\begin{proof}
Notice that $\bar{A}_{ij}^2 = \sum_{k=1}^n \bar{A}_{ik} \bar{A}_{kj}$ is a sum of $n$ independent random variables between 0 and 1. Since $E[\bar{A}_{ij}^2] = P_{ij}^2$, by Hoeffding's inequality, we have
\[
	\mathbb{P}\left( |A_{ij}^2 - P_{ij}^2| > \sqrt{2 n \log n} \right)
    \le 2 \exp \left( - \frac{4 n \log n}{n} \right)
    = \frac{2}{n^4}.
\]
Using a union bound, we have
\begin{align*}
	\mathbb{P} \left( \|A^2 - P^2\|_F > \sqrt{2 n^3 \log n} \right)
    & \le \mathbb{P} \left( \cup_{i,j} \left\{ |A_{ij}^2 - P_{ij}^2| > \sqrt{2 n \log n} \right\} \right) \\
    & \le \sum_{i,j} \mathbb{P} \left( |A_{ij}^2 - P_{ij}^2| > \sqrt{2 n \log n} \right) \\
    & = \sum_{i,j} \frac{2}{n^4}
    = \frac{2}{n^2}.
\end{align*}
\end{proof}








\begin{proposition}
\label{prop:eigengap}
Assuming that the matrix $\mathbb{E}[X X^T]$ is rank $d$ and has distinct eigenvalues. Also we suppose there exists $\delta > 0$ such that $\delta < \min_{i \ne j} |\lambda_i(\mathbb{E}[X X^T]) - \lambda_j(\mathbb{E}[X X^T])|$ and $\delta < \lambda_d(\mathbb{E}[X X^T]$. For $i, j \le d+1$, $i \ne j$ and $n$ sufficiently large, it holds with probability greater than $1 - \frac{2d^2}{n^2}$ that
\[
	|\lambda_i(P) - \lambda_j(P)| > \delta n / 2.
\]
\end{proposition}
\textbf{Remark:} This is Proposition 4.3 in \href{url}{http://arxiv.org/pdf/1207.6745.pdf}.





\begin{lemma}
\label{lm:Snorm}
Let $(X, A^{(1)}, \cdots, A^{(m)}) \sim \mathrm{RDPG}(F)$ and let $\bar{X} = \bar{V} \bar{S}^{1/2}$ be our estimate for $X$ based on $\bar{A} = \frac{1}{m} \sum_{t = 1}^m A^{(t)}$.
Define $\delta_d = \min \left\{ \min_{1 \le i \le d-1} |\lambda_i - \lambda_{i+1}|, \lambda_d \right\}$ where $\lambda_1 > \lambda_2 > \cdots > \lambda_d$ are the ordered, distinct, positive eigenvalues of $\Delta$. For $n$ sufficiently large, following bounds hold with probability greater than $1-\eta$,
\[
	\frac{\delta_d n}{2} \le \|S\| \le n,
\]
\[
	\frac{\delta_d n}{2} \le \|\bar{S}\| \le n.
\]
\end{lemma}
\textbf{Remark:} This is an extended version of Lemma 1 in \textbf{\textit{A limit theorem for scaled eigenvectors of random dot product graphs}}. \\
\noindent
\begin{proof}
To be continue.
\end{proof}








\begin{lemma}
\label{lm:AllDiff}
Let $(X, A^{(1)}, \cdots, A^{(m)}) \sim \mathrm{RDPG}(F)$ and let $\bar{X} = \bar{V} \bar{S}^{1/2}$ be our estimate for $X$ based on $\bar{A} = \frac{1}{m} \sum_{t = 1}^m A^{(t)}$.
Define $\delta_d = \min \left\{ \min_{1 \le i \le d-1} |\lambda_i - \lambda_{i+1}|, \lambda_d \right\}$ where $\lambda_1 > \lambda_2 > \cdots > \lambda_d$ are the ordered, distinct, positive eigenvalues of $\Delta$. Let $c$ be arbitrary. There exists a constant $n_0(c)$ such that if $n > n_0$, then for any $\eta$ satisfying $n^{-c} < \eta < 1/2$, the following bounds hold with probability greater than $1-\eta$,
\begin{equation}
\label{eqn:AbarDiff}
\|X X^T - \bar{A}\| \le 4(n \log(n/\eta))^{1/2},
\end{equation}
\begin{equation}
\label{eqn:VDiff}
\|\bar{V} - V\|_F \le 8 \delta_d^{-1} n^{-1/2} (2d \log(n/\eta))^{1/2}.
\end{equation}
\begin{equation}
\label{eqn:XDiff}
\|\bar{X} - \tilde{X}\|_F = \|\bar{V} \bar{S}^{1/2} - V S^{1/2}\|_F \le 6 \delta_d^{-1} (2d \log(n/\eta))^{1/2},
\end{equation}
\end{lemma}
\textbf{Remark:} This is an extended version of Lemma 1 in \textbf{\textit{A limit theorem for scaled eigenvectors of random dot product graphs}}. \\
\noindent
\begin{proof}
By Theorem \ref{thm:AbarDiff}, we get Inequality (\ref{eqn:AbarDiff}).



For Inequality (\ref{eqn:VDiff}), by Davis-Kahan Theorem, Proposition \ref{prop:eigengap} and Inequality (\ref{eqn:AbarDiff}), we have
\begin{align*}
	\| \bar{V}_i - V_i \|_2
	& \le \frac{\|\bar{A} - P\|_2}{\min \{ |\lambda_i(P) - \lambda_{i+1}(P)|, |\lambda_i(P) - \lambda_{i-1}(P)| \}} \\
    & \le \frac{\|\bar{A} - P\|_2}{\delta_d n/2} \\
    & \le 8 \delta_d^{-1} n^{-1/2} (\log(n / \eta))^{1/2},
\end{align*}
where $\bar{V}_i$ represents the $i$th column of $\bar{V}$ and $V_i$ represents the $i$th column of $V$.
\todo[inline, color=green!40]{It is not very clear here since Proposition \ref{prop:eigengap} happens with probability $1- 2d^2/n^2$.}
Thus
\begin{align*}
	\|\bar{V} - V\|_F & = \left( \sum_{i=1}^d \| \bar{V}_i - V_i \|_2^2 \right)^{1/2} \\
    & \le \left( d \cdot \left(8 \delta_d^{-1} n^{-1/2} (\log(n / \eta))^{1/2} \right)^2 \right)^{1/2} \\
    & = 8 \delta_d^{-1} n^{-1/2} (d \log(n / \eta))^{1/2}
\end{align*}



For Inequality (\ref{eqn:XDiff}), first notice that
\begin{align*}
	& \|\bar{V} \bar{S}^{1/2} - V S^{1/2} \|_F \\
    \le & \| \bar{V} \bar{S}^{1/2} - \bar{V} S^{1/2} \|_F + \| \bar{V} S^{1/2} - V S^{1/2} \|_F \\
    = & \| \bar{V} ( \bar{S}^{1/2} - S^{1/2} ) \|_F + \| ( \bar{V} - V) S^{1/2} \|_F.
\end{align*}

By Proposition \ref{prop:AbarSquareDiff}, we know
\begin{align*}
	\lambda_i^2(|\bar{A}|) - \lambda_i^2(P) = & \lambda_i(\bar{A}^2) - \lambda_i(P^2) \\
    \le & \lambda_i(\bar{A}^2 - P^2) why this is true? \\
    \le & \| \bar{A}^2 - P^2 \|_2 \\
    \le & \| \bar{A}^2 - P^2 \|_F \\
    \le & \sqrt{2 n^3 \log n}
\end{align*}
with probability greater than $1 - \frac{2}{n^2}$.
Thus it holds with probability greater than $1 - \frac{2}{n^2}$ that
\begin{align*}
	& \lambda_i^{1/2}(|\bar{A}|) - \lambda_i^{1/2}(P) \\
    = & \frac{\lambda_i^2(|\bar{A}|) - \lambda_i^2(P)}{\left(\lambda_i(|\bar{A}|) + \lambda_i(P)\right)\left(\lambda_i^{1/2}(|\bar{A}|) + \lambda_i^{1/2}(P)\right)} \\
    \le & \frac{\sqrt{2 n^3 \log n}}{(\delta_d n)^{3/2}}
    = \sqrt{\frac{2 \log n}{\delta_d^3}}.
\end{align*}
Combined with Inequality (\ref{eqn:VDiff}) and Lemma \ref{lm:Snorm}, we have
\begin{align*}
	& \|\bar{V} \bar{S}^{1/2} - V S^{1/2} \|_F \\
    \le & \| \bar{V} ( \bar{S}^{1/2} - S^{1/2} ) \|_F + \| ( \bar{V} - V) S^{1/2} \|_F \\
    \le & \|\bar{V}\|_F \|\bar{S}^{1/2} - S^{1/2}\|_2 + \| \bar{V} - V\|_F \|S^{1/2}\|_2 \\
    \le & \sqrt{d} \cdot \sqrt{\frac{2 \log n}{\delta_d^3}} + 8 \delta_d^{-1} n^{-1/2} (2d \log(n/\eta))^{1/2} \cdot \sqrt{n} \\
    = & 16 \delta_d^{-1} (2d \log(n/\eta))^{1/2}.
\end{align*}
\todo[inline, color=green!40]{Question: I assumed $\delta_d \ge 1$ here.}
\end{proof}









\begin{lemma}
\label{lm:SDiff}
In the setting of Lemma \ref{lm:AllDiff}, with probability greater than $1-2\eta$,
\[
	\|S - \bar{S}\|_F \le C \delta_d^{-2} d \log(n/\eta).
\]
\end{lemma}
\textbf{Remark:} This is an extended version of Lemma 2 in \textbf{\textit{A limit theorem for scaled eigenvectors of random dot product graphs}}. \\
\noindent
\begin{proof}
Replace $A$ with $\bar{A}$. Omitted here.
\end{proof}








\begin{lemma}
\label{lm:VVhatDiff}
In the setting of Lemma \ref{lm:AllDiff}, with probability greater than $1-2\eta$,
\[
	\|V^T \bar{V} - I \|_F \le \frac{C d \log(n/\eta)}{\delta_d^2 n}.
\]
\end{lemma}
\textbf{Remark:} This is an extended version of Lemma 3 in \textbf{\textit{A limit theorem for scaled eigenvectors of random dot product graphs}}. \\
\begin{proof}
Replace $A$ with $\bar{A}$. Omitted here.
\end{proof}







\begin{theorem}
\label{thm:clt}
Let $(X, A^{(1)}, \cdots, A^{(m)}) \sim \mathrm{RDPG}(F)$ be $d$-dimensional random dot product graph, i.e., $F$ is a distribution for points in $\mathbb{R}^d$, and let $\bar{X}$ be our estimate for $X$ based on $\bar{A} = \frac{1}{m} \sum_{t = 1}^m A^{(t)}$. We assume $F$ has a diagonal second moment matrix with distinct positive entries along the diagonal. Let $\Phi(z, \Sigma)$ denote the cumulative distribution function for the multivariate normal, with mean zero and covariance matrix $\Sigma$, evaluated at $z$. Then there exists a sequence of orthogonal matrices $W_n$ converging to the identity almost surely such that for each component $i$ and any $z \in \mathbb{R}^d$,
\[
	\mathbb{P}\{ \sqrt{n}(W_n \bar{X}_i - X_i) \le z \} \rightarrow \int \Phi(z, something \delta^{-2} \Sigma(x)) dF(x)
\]
where $\bar{\Sigma}(x) = something \frac{1}{m} \Delta^{-1} E \left[ X_j X_j^T (x^T X_j - (x^T X_j)^2 ) \right) \Delta^{-1}$ and $\Delta = E[X_1 X_1^T]$ is the second moment matrix. That is, the sequence of random variables $\sqrt{n}(W_n \bar{X}_i - X_i)$ converges in distribution to a mixture of multivariate normals. We denote this mixture by $\mathcal{N}(0, something \delta^{-2} \Sigma^2(X_i))$.
\end{theorem}
\textbf{Remark:} This is an extended version of Theorem 1 in \textbf{\textit{A limit theorem for scaled eigenvectors of random dot product graphs}}. \\
\begin{proof}
Let $\bar{A} V S^{-1/2} = [Y_1, \cdots, Y_n]^T$ where $Y_i \in \mathbb{R}^d$. We first show that $\sqrt{n} (Y_i - \tilde{X}_i)$ is multivariate normal in the limit. We have
\begin{align*}
	\sqrt{n}(Y_i - \tilde{X}_i) & = \sqrt{n} \left( (\bar{A} \tilde{X} S^{-1})^T_{\cdot i}  - (P \tilde{X} S^{-1})^T_{\cdot i}\right) \\
    & = n S^{-1} W_n \left( \frac{1}{\sqrt{n}} \sum_{j=1}^n (\bar{A}_{ij} - P_{ij}) X_j \right) \\
    & = n S^{-1} W_n \left( \frac{1}{\sqrt{n}} \sum_{j \ne i}(\bar{A}_{ij} - X_i^T X_j)X_j - \frac{X_i^T X_i}{\sqrt{n}} X_i \right).
\end{align*}
Conditioning on $X_i = x_i$, the scaled sum
\[
	\frac{1}{\sqrt{n}} \sum_{j \ne i}(\bar{A}_{ij} - X_i^T X_j)X_j
\]
is a sum of i.i.d, mean zero random variables with covariance matrix
\[
	\tilde{\Sigma}(x) = something \frac{1}{m} E \left[ X_j X_j^T (x^T X_j - (x^T X_j)^2 ) \right).
\]
\end{proof}































\newpage

The extended version of Theorem 3.3 is that:

\begin{theorem}
\label{thm:clt}
Let $(X, A^{(1)}, \cdots, A^{(m)}) \sim \mathrm{RDPG}(F)$ and let $\bar{X}$ be our estimate for $X$ based on $\bar{A} = \frac{1}{m} \sum_{t = 1}^m A^{(t)}$. Let $\Phi(z, \sigma^2)$ denote the normal cumulative distribution function, with mean zero and variance $\sigma^2$, evaluated at $z$. Then for each component $i$ and any $z \in \mathbb{R}$,
\[
	P\{ \sqrt{n}(\bar{X}_i - X_i) \le z \} \rightarrow \int \Phi(z, \delta^{-2} \sigma^2(x_i)) dF(x_i)
\]
where $\sigma^2(x_i) = \frac{1}{m} \left( x_i E[X_1^3] + x_i^2 E[X_1^4] \right)$ and $\delta = E[X_1^2]$. That is, the sequence of random variables $\sqrt{n}(\bar{X}_i - X_i)$ converges in distribution to a mixture of normals. We denote this mixture by $\mathcal{N}(0, \delta^{-2} \sigma^2(X_i))$.
\end{theorem}
\begin{proof}
Proposition \ref{prop:Y} is extended version for Proposition 3.5 in Avanti's paper. And the rest argument would be exactly the same assuming Proposition 3.2 in Avanti's paper is true for $\bar{A}$.
\end{proof}


\begin{proposition}
\label{prop:Y}
Let $Y = \hat{\lambda}^{-1/2} \bar{A} V$ be the vector in $\mathbb{R}^n$ with components $Y_i = \bar{\lambda}^{-1/2} \sum_{j=1}^n \bar{A}_{ij} V_j$.
Taking $X_i = x_i$ as given, we have
\[
	\sqrt{n}(Y_i - \frac{}{}x_i) \stackrel{\mathcal{L}}{\rightarrow} \mathcal{N}(0, \delta^{-2} \sigma^2(x_i))
\]
where $\sigma^2(\cdot)$ and $\delta$ are as in Theorem \ref{thm:clt}.
\end{proposition}
\begin{proof}
Observe that
\[
	\sqrt{n}(Y_i - \frac{}{}x_i) = \frac{n}{\hat{\lambda}^{1/2}\|X\|} \cdot \frac{1}{\sqrt{n}} \left[ \left( \sum_{j \ne i} (\bar{A}_{ij} - x_i X_j) X_j ) \right) - x_i^3 \right].
\]
The scaled sum
\[
 	\frac{1}{\sqrt{n}} \left( \sum_{j \ne i} (\bar{A}_{ij} - x_i X_j) X_j  \right)
\]
is a sum of independent, identically distributed random variables each with mean zero and variance
\[
	\sigma^2(x_i) = \frac{1}{m} x_i E[X_1] + \frac{m-1}{m} x_i^2 E[X_1^2] - x_i^2 E[X_1^4].
\]

Then the rest would be the same as Proposition 3.5 in Avanti's paper.

Detailed calculation for the expectation is as following:
\begin{align*}
	& E \left[\frac{1}{\sqrt{n}} \left( \sum_{j \ne i} (\bar{A}_{ij} - x_i X_j) X_j \right) \right] \\
    = & \sum_{j \ne i} E\left[\frac{1}{\sqrt{n}} (\bar{A}_{ij} - x_i X_j) X_j \right] \\
    = & \sum_{j \ne i} \frac{1}{\sqrt{n}} E\left[ \frac{1}{m} \sum_{t = 1}^m A^{(t)}_{ij} X_j - x_i X_j^2 \right]
\end{align*}
And
\begin{align*}
	& E\left[ \frac{1}{m} \sum_{t = 1}^m A^{(t)}_{ij} X_j - x_i X_j^2 \right] \\
    = & E\left[ E\left[ \left( \frac{1}{m} \sum_{t = 1}^m A^{(t)}_{ij} X_j - x_i X_j^2\right) | X_j \right] \right] \\
    = & E\left[ \frac{1}{m} \sum_{t = 1}^m A^{(t)}_{ij} X_j - x_i X_j^2 \right] \\
    = & E\left[  \frac{1}{m} \sum_{t = 1}^m x_i X_j^2 - x_i X_j^2 \right] \\
    = & 0
\end{align*}
since $A_{ij}^{(t)}$ is Bernoulli with probability $x_i X_j$.

Similarly, detailed calculation for the variance is as following:
\begin{align*}
	& Var \left( (\bar{A}_{ij} - x_i X_j) X_j \right) \\
    = & Var \left(\bar{A}_{ij} X_j - x_i X_j^2 \right) \\
    = & E \left[ \left( \bar{A}_{ij} X_j - x_i X_j^2 \right)^2 \right] \\
    = & E \left[ \bar{A}_{ij}^2 X_j^2 + x_i^2 X_j^4 - 2 x_i X_j^3 \bar{A}_{ij} \right] \\
    = & E \left[ E \left[ \bar{A}_{ij}^2 X_j^2 + x_i^2 X_j^4 - 2 x_i X_j^3 \bar{A}_{ij} | X_j \right] \right] \\
    = & E \left[ Var(\bar{A}_{ij} X_j) + E[\bar{A}_{ij} X_j]^2 - x_i^2 X_j^4 \right] \\
    = & E \left[ \frac{1}{m^2} \sum_{t=1}^m Var(A_{ij}^{(t)} X_j) + x_i^2 X_j^4 - x_i^2 X_j^4 \right] \\
    = & E \left[ \frac{1}{m^2} \sum_{t=1}^m (x_i X_j^3 - x_i^2 X_j^4) \right] \\
    = & \frac{1}{m} (x_i X_j^3 - x_i^2 X_j^4).
\end{align*}



\end{proof}












\end{document}