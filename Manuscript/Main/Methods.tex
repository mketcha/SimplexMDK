\section{Methods}
	\subsection{Algorithm: $\hat{P}$}
	\textbf{Input:} $A_1, A_2,..., A_M$, with each $A_i \in \{0,1\}^{N \times N}$ having vertex correspondence
	\begin{enumerate}
		\item Calculate $\bar{A} = \frac{1}{M}\sum\limits_{m = 1}^M A_m$
		\item Estimate SBM parameter $k$ (see section 5.2)
		\item Form the matrix $X \in R^{n \times k}$ with the columns in $X$ consisting of the eigenvectors corresponding to the largest, in absolute value, eigenvalues of $\bar{A}$, with the diagonal entries augmented (see section 5.3).
		\item  $\hat{P} = XX^{T}$
	\end{enumerate}	
	\subsection{Choosing Dimension}
	Often in dimensionality reduction techniques, the choice for dimension, $k$, relies on visually analyzing a plot of the ordered eigenvalues, looking for a "gap" or "elbow" in this scree-plot.  Zhu and Ghodsi \cite{Zhu2006} present an automated method for finding this gap in the scree-plot that takes only the ordered eigenvalues as an input.  In order to prevent underestimating $k$, which is much more harmful than over-estimating, we initialize $k_0 = 0$ and iterate over the Zhu and Ghodsi algorithm by removing the first $k_{i-1}$ eigenvalues from calculation at each iteration to determine the location of the "next elbow".  For the experiments performed in this work, we choose $k$ to be $k_3$ under this approach.
	
	(Show the scree plot for a connectome here)
	
	\subsection{Graph Diagonal Augmentation}
	The graphs examined in this work are hollow, in that there are no self-loops and thus the diagonal entries of the adjacency matrix are 0.  This creates a bias in the calculation of the eigenvectors.  We minimize this bias by using an iterative method developed by Scheinerman and Tucker \cite{Scheinerman2010}. In this method, steps 3 and 4 of the $\hat{P}$ algorithm are repeated, each time replacing the diagonal component of $\bar{A}$ with the diagonal of $\hat{P}$, until $\hat{P}$ converges.
	\subsection{Dataset Description}
	The connectomes analyzed were created from resting state functional MRI (fMRI) and Diffusion Tensor Imaging (DTI) scans from the Consortium for Reliability and Reproducibility (CoRR) and are available via the International Neuroimaging Data-sharing Initiative (INDI).  The SWU 4 - Southwest University image collection was used to generate 464 connectomes with 788 anatomically corresponding vertices.  (Need to describe how graphs were made with reference, will ask eric bridgeford)
	\subsection{Source code and data}
	\subsection{Proof of Var($\hat{P}$)}
	We provide an outline of the proof for the Var($\hat{P}$) result in section 3.1.  The full proof is provided at ...